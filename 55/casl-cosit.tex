\documentclass[oribibl]{llncs}

%%% Base Packages
\usepackage[latin1]{inputenc} 
%%\usepackage[active]{srcltx}
\usepackage{stmaryrd,mathbbol,mathrsfs}
\usepackage{amsmath,amstext,amssymb,amsopn}
\usepackage{exscale,calc,ifthen,array}
\usepackage{float}
\usepackage{url}
\usepackage[sectionbib,numbers]{natbib}
%%\bibliographystyle{apalike}
\bibliographystyle{abbrvnat}
\renewcommand{\refname}{References}
\renewcommand{\bibname}{References}

%%% sjw Packages
\usepackage{casl-cosit}

\usepackage{xspace}
\newcommand{\CASL}{\textsc{Casl}\xspace }
\newcommand{\CoFI}{\textmd{\textsc{CoFI}}\xspace }
\newcommand{\HetCASL}{\textmd{\textsc{HetCasl}}\xspace }
\newcommand{\ModalCASL}{\textmd{\textsc{ModalCasl}}\xspace }
\newcommand{\HasCASL}{\textsc{HasCasl}\xspace }
\newcommand{\Hets}{\textmd{\textsc{Hets}}\xspace }


\usepackage{listings}
\lstset{stringstyle=\rmfamily,
  showstringspaces=false,
  flexiblecolumns,
  mathescape=true,
  basicstyle=\small,
  keywords={and,end,fit,forall,free,hide,library,logic,op,ops,pred,preds,
    reveal,sort,spec,then,to,type,version,view,with,local,within},
  otherkeywords={\%implies,\%def,\%mono,\%cons},
  emph={AlgebraOfBinaryRelations,AllenIA,AtomicBooleanAlgebra,AtomicRelationAlgebra,
    BaseRelations,BinaryRelations,BooleanAlgebra,BooleanAlgebraWithCompl,
    CompositionTable,
    EuclideanPlane,
    ExtBooleanAlgebra,ExtBooleanAlgebraByPartialOrder,ExtRelationAlgebraByPartialOrder,
    FullAlgebraOfBinaryRelations,
    CompClosedBaseRelModel,
    ConstructRelationAlgebra,
    ConstructPreModelFromBaseRelModel,RelationsFromBaseRelModel,ConstructModel,
    GoodCompositionTable,
    HasCASL,
    JEPDBaseRelModel,
    MetricSpace,
    PartialOrder,PreOrder,AntisymmetricRelation,
    RelationAlgebra,RCC5,RCC5BaseRelations,RCC5CompositionTable,
    RCC5OpenDiscModel,RCC5OpenDiscBaseRelModel,
    Set,SetRepresentationOfRelations,SetAlgebraOfBinaryRelations},
  emphstyle=\sc,
  escapechar=\@,
  linewidth=\floatwidth,
  xleftmargin=2em,
  xrightmargin=-2em}

\def\SpecName#1{\text{\sc{#1}}}
\def\cmps{\mathbin{\circ}}
\let\eps\in
\let\isIn\in
\def\compl{-}
\def\inv#1{#1^{\smallsmile}}
\def\casldot{\;{\scriptstyle\bullet}\;}
\def\name#1{\text{\it #1}}
\def\OPdr{\name{dr}}
\def\OPec{\name{ec}}
\def\OPdc{\name{dc}}
\def\OPpo{\name{po}}
\def\OPpp{\name{pp}}
\def\OPppi{\name{ppi}}
\def\OPtpp{\name{tpp}}
\def\OPtppi{\name{tppi}}
\def\OPntpp{\name{ntpp}}
\def\OPntppi{\name{ntppi}}
\def\OPeq{\name{eq}}
\def\OPb{\name{b}}
\def\OPbi{\name{bi}}
\def\OPm{\name{m}}
\def\OPmi{\name{mi}}
\def\OPo{\name{o}}
\def\OPoi{\name{oi}}
\def\OPs{\name{s}}
\def\OPsi{\name{si}}
\def\OPf{\name{f}}
\def\OPfi{\name{fi}}
\def\OPd{\name{d}}
\def\OPdi{\name{di}}
\def\OPe{\name{e}}
\def\OPrep{\name{rep}}
\def\OPid{\name{id}}
\def\OPBaseRel{\name{BaseRel}}
\def\OPRel{\name{Rel}}
\def\OPRelation{\name{Relation}}
\def\OPElem{\name{Elem}}
\def\OPAtomElem{\name{AtomElem}}
\def\OPAtomRel{\name{AtomRel}}
\def\OPdrRel{\name{drRel}}
\def\OPpoRel{\name{poRel}}
\def\OPeqRel{\name{eqRel}}
\def\OPppRel{\name{ppRel}}
\def\OPppiRel{\name{ppiRel}}


%%\def\union{\mathbin{union}}
%%\def\intersection{\mathbin{intersection}}
%%\def\ssubset{\mathbin{subset}}
\let\union\cup
\let\intersection\cap
\let\ssubset\sqsubset
\let\emptySet\emptyset
\newif\ifextended
\extendedfalse

%%% PDF switch 
\newif\ifpdf
  \ifx\pdfoutput\undefined
  \pdffalse    % we are not running PDFLaTeX
  \else
  \pdfoutput=1    % we are running PDFLaTeX
  \pdftrue
\fi

%%% graphics

\ifpdf 
   \usepackage[pdftex]{graphicx}
\else
   \usepackage{graphicx}
\fi



%%% fonts  
\usepackage{pslatex}


\newcommand{\pp}[2][2.5em]{\parbox{#1}{\centering #2}}
\renewcommand{\P}{{\sf P}} 
\newcommand{\NP}{{\sf NP}} 
\newcommand{\EXPTIME}{{\sf EXPTIME}} 




\begin{document}

%%%
\title{%
  \CASL Specifications of Qualitative Calculi}
\author{Stefan W\"olfl\inst{1}\and Till Mossakowski\inst{2}}
\institute{Department of Computer Science, University of Freiburg,
  Georges-K\"ohler-Allee, 79110~Freiburg, Germany, \email{woelfl@informatik.uni-freiburg.de}
  \and 
  Department of Computer Science, University of Bremen, P.O. Box 330440,\\ 28334~Bremen, Germany,
  \email{till@tzi.de}
}


\maketitle


\begin{abstract}
  In AI a large number of calculi for efficient reasoning about
  spatial and temporal entities have been developed.  The most
  prominent temporal calculi are the point algebra of linear time and
  Allen's interval calculus.  Examples of spatial calculi include
  mereotopological calculi, Frank's cardinal direction calculus,
  Freksa's double cross calculus, Egenhofer and Franzosa's
  intersection calculi, and Randell, Cui, and Cohn's region connection
  calculi.

  These calculi are designed for modeling specific aspects of space or
  time, respectively, to the effect that the class of intended models
  may vary widely with the calculus at hand. But from a formal point
  of view these calculi are often closely related to each other.
%%, even across the border of the space-time dichotomy. 
  For example, the spatial region connection calculus RCC5 may be
  considered a coarsening of Allen's (temporal) interval calculus. And
  vice versa, intervals can be used to represent spatial objects that
  feature an internal direction.

  The central question of this paper is how these calculi as well as
  their mutual dependencies can be axiomatized by algebraic
  specifications. This question will be investigated within the
  framework of the \emph{Common Algebraic Specification Language}
  (\CASL), a specification language developed by the \emph{Common
    Framework Initiative for algebraic specification and development}
  (\CoFI). We explain scope and expressiveness of \CASL by discussing
  the specifications of some of the calculi mentioned before.
\end{abstract}


\section{Introduction: Calculemus!} 
  
In the past 25 years qualitative spatial and temporal reasoning has
evolved to a discipline in its own right within AI. Qualitative
reasoning aims at describing the common-sense background knowledge on
which our human perspective on the physical reality is based. The
calculi, that is, formal languages and reasoning techniques, developed
in this research area are of special interest for all application
fields that rely on human-machine interaction in static or dynamically
changing spatial environments.  For example, some of these calculi may
be implemented for handling spatial GIS queries efficiently and some
may be used for navigating, and communicating with, a mobile robot.

One will hardly find an exact definition of the notion of qualitative
reasoning, but the whole research area has been very much inspired by
Hayes' na\"ive manifesto \citep{hayes-a:78-naive}. In fact, in
different areas of mathematics and physics very expressive formalisms
for reasoning about space and time have been developed. But from a
computer scientist's point of view most of these formalisms are too
expressive. Since there is an inevitable trade-off between the
expressiveness of a language and the computational costs for reasoning
with its formulae, expressiveness can get a crucial point when these
calculi are to be integrated in applications. Thus the fundamental
idea of qualitative reasoning is to restrict the vocabulary of rich
mathematical theories in such a way that diversified aspects of these
theories are treated within distinguished decidable fragments with
simple qualitative (\ie\ non-metrical) languages.

From this starting point a large number of calculi for efficient
reasoning about spatial and temporal entities have been proposed in
the literature.  The most prominent temporal calculi are the so-called
point algebra, which deals with instants of a given linear flow of
time, and Allen's interval algebra \citep{allen-a:83-maintaining},
which describes possible relations between intervals in linear
flows of time (cf.~Fig.~\ref{fig:AllenRelations}).  Analogous calculi
have been proposed for more general classes of models such as
branching flows of time \citep[e.\,g.,][]{broxvall-a:01-point} or even
structures, where flows of time are just required to satisfy the
conditions of a partial order
\citep[e.\,g.,][]{broxvall-a:99-towards}. Despite these one-sorted
calculi, also many-sorted calculi have been proposed.  For example,
Vilain's point-interval calculus \citep{vilain-a:82-system} deals with
instants and intervals in linear flows of time and may be considered a
combination of the point algebra and the interval algebra.
\begin{figure}
\begin{center}
  \includegraphics[width=80mm]{graphics/allen}
\end{center}
\caption{Allen's interval relations}
\label{fig:AllenRelations}
\end{figure}

%% In this paper we will mainly focus on \emph{spatial} qualitative
%% calculi.  
Examples of spatial calculi include mereotopological calculi (e.\,g.,
\citep{bennett-b:97-logical}), Frank's cardinal direction calculus
\citep{frank-a:91-qualitative}, Freksa's double cross calculus
\citep{freksa-a:92-using}, Egenhofer and Franzosa's 4- and
9-intersection calculi
\citep{egenhofer-a:91-reasoning,egenhofer-a:91-point}, Ligozat's
flip-flop calculus \citep{ligozat-a:93-qualitative}, and various
region connection calculi proposed by
\citet{randell-a:92-spatial,cohn-a:97-rcc},
\citet{duentsch-a:99-relation}, and \citet{gerevini-a:98-combining}.
It is interesting to see that even these few calculi employ concepts
from a wide range of mathematical theories. Some of them are based on
geometrical notions such as lines, half-planes, and angels, some
describe relations between physical objects in terms of point set
topology, and some include qualitative size information.

Interestingly, some spatial calculi are closely related to the
temporal calculi mentioned previously.  For example, the \emph{2-point
  calculus} describes points of the plane and their relationships in
terms of the point-to-point relations between their coordinates. This
means, that one considers for each dimension one of the three possible
point-to-point relations $<$, $=$, and $>$ between the point
coordinates. The relation $(=,>)$, for instance, expresses the relation
``north-to'', $(>,>)$ corresponds to ``north-east-to'', etc. For this
reason the 2-point calculus is often referred to as \emph{cardinal
  direction calculus} in the literature.
\begin{figure}[ht]
  \begin{equation*}
    \includegraphics[width=35mm]{graphics/cardinaldirection}
    \qquad\qquad \includegraphics[width=45mm]{graphics/rectangle-gray}
  \end{equation*}
  \caption{Spatial calculi derivable from temporal algebras: The cardinal 
    direction calculus and the rectangle calculus}
  \label{fig:CardinalDirectionCalculus}
\end{figure}
Analogously, the \emph{rectangle calculus} describes possible
relations between rectangles in the plane by comparing their 
coordinate projections in terms of the interval algebra.


\smallskip


To sum up this little discussion, researchers in the domain of
qualitative reasoning face a vast, still increasing amount of calculi
(much more than the rather incomplete list of calculi previously
mentioned can indicate). On the other hand, many of these calculi are
closely related to each other: some are simple extensions of others,
some show similarities on the syntactic level, some have related
classes of intended models, \ie\ they are based on more or less the
same background theory. Thus the guiding question of this paper is how
to present qualitative calculi within a common framework in such a way
that the mutual dependencies between them and their respective
background theories becomes more transparent.  In fact, transparency
is an important issue for both avoiding redundancies and ensuring
reusability of these calculi.

In our opinion, a na�ve ontological classification system of
qualitative or other calculi will not be able to fulfill these
requirements in an adequate manner. A calculus classified as, say,
``temporal'' will always be a calculus about \emph{temporal entities}
such as instants or intervals, and can hardly be subsumed under the
term ``spatial calculus''. Moreover, connections between temporal and
spatial calculi like the ones previously mentioned seem inexpressible
in any ontology of such calculi.  For this reason we propose to
present qualitative calculi by means of algebraic specification, as
was already suggested by \citet{frank-a:99-one}.  More exactly, we
explain how to develop such specifications within the \emph{Common
  Algebraic Specification Language} (\CASL).
%, a specification
% language developed by the \emph{Common Framework Initiative for
%  algebraic specification and development (\CoFI)}.


\smallskip

The paper is organized as follows: In section~2 we briefly
explain fundamental notions related to qualitative reasoning in more
detail.  Section~3 provides a short introduction into \CASL and its
extension \HasCASL.  Then in section~4 we discuss algebraic
specifications of the concepts introduced in section~2 in an informal
manner.


  
\section{Qualitative Calculi}

To exemplify the most fundamental ideas of qualitative reasoning, let
us discuss the point algebra for linear time in more detail.  From a
model-theoretical perspective this point algebra aims at describing
the class of \emph{linear flows of time}, \ie\ first order structures
$\FoT=\tupel{T,<}$ that are models of the following
axioms:%
%%\footnote{Here and it what follows we will not sharply distinguish the
%%  symbols used in the object language from those in the meta language,
%%   as far as is this is well understood from the context.}
\begin{description}
\item[Irreflexivity:] $\fa x\, x\not<x$
\item[Transitivity:] $\fa xyz (x<y\lu y<z \li x< z)$
\item[Linearity:] $\fa xy (x<y \lo x=y \lo y<x)$
\end{description}
From an ontological point of view, the point algebra takes instants of
time as \emph{primary objects}, and states that these entities are
linearly ordered. If we shift consideration from these primary
objects towards the relations between them, we can state the following
observations: From $<$ being irreflexive and transitive, it follows
that the relations $<$, $=$, and $>$ (where $>$ is just defined as the
converse of relation $<$) are pairwise disjoint.  Linearity guarantees
that these relations are jointly exhaustive, \ie\ for each pair of
instants $t$ and $t'$, one of the relations $t<t'$, $t=t'$, or $t>t'$
holds.  Speaking algebraically, the set $\{<,=,>\}$ forms a 
jointly exhaustive and pairwise disjoint (JEPD)
system of relations. 


Often temporal or spatial information is imprecise, for example, when
we only have the information that instant $t$ is not before instant
$t'$, or that instants $t$ and $t'$ are distinct. In this situation it
becomes interesting to consider not only the \emph{base relations}
$<$, $=$, and $>$, but also arbitrary unions of them.  Obviously, the
system of unions of base relations defines an atomic Boolean
algebra with the base relations as atoms. By the way,
since the system of base relations is JEPD, unions of base relations
may also be represented as \emph{sets} of base relations.

Reasoning problems, then, are usually formulated as constraint
satisfaction problems.  A constraint network is a finite set of
constraints where each constraint is a formula of the form $x\, R\, y$
with variables $x$ and $y$ (taking values in given domains $D_x$ and
$D_y$) and a relation $R$ (a set of base relation) defined between
the domains of $x$ and $y$. Typical reasoning tasks are then to
determine whether a constraint set is satisfiable, to check that some
constraint is entailed by a constraint network, and to compute an
equivalent minimal constraint set\,---\,it is not hard to see that all
these reasoning tasks are equivalent under Turing reductions.

A crucial aspect for developing efficient algorithms for qualitative
spatial and temporal reasoning is the fact that the underlying model
classes usually contain infinite models. Hence, in order to test
satisfiability of constraint networks, it is not feasible to enumerate
all models until one finds a satisfying model. For this reason other
techniques (such as path-consistency algorithms) must be applied for
testing satisfiability. Many of these techniques, in turn, rely on
semantically verified composition tables that list which relations are
consistent when two base relations are composed.  For example,
Table~\ref{tbl:comp-PAlin} presents the composition table of the point
algebra for linear time.%
\footnote{%
  Note that there are (at least) two ways of reading such composition
  tables, the \emph{extensional} and the \emph{consistency-based}
  reading \citep[cf.][]{bennett-a:97-when}.  Following, we will only
  use the extensional reading, which means that the algebraic function
  of composing relations (as used, for example, in relation algebras)
  coincides with its set-theoretical characterization. In the case of
  the point algebra, for example, the consistency-based reading is
  correct for the class of linear flows of time, while the extensional
  reading is only correct for the class of dense linear flows of
  time without endpoints.}
\begin{table}[ht]
  \small
  \caption{The composition table of the point algebra for 
    linear time}
  \label{tbl:comp-PAlin}
  \begin{equation*}
\begin{array}{c|ccc}
    & \enskip{<} \enskip\strut& \enskip >\enskip\strut &\enskip =\enskip\strut \\\cline{1-4}\\[-1ex]
\enskip{<}\enskip\strut   & < & <,=,> & < \\[1ex]
>   &\, <,=,> & > & > \\[1ex]
=   & < & > & =
\end{array}
\end{equation*}
%%\begin{legende}[90mm] 
%%   The symbol 1 is used to denote the set/union of all base relations.
%%\end{legende}
\end{table}
But a set of base relations together with a composition table
satisfying some minimal conditions defines a \emph{relation algebra}
on the set of all unions of base relations. For this reason studying
relation algebras has become a central aspect in the field of
qualitative reasoning.





\section{\CASL and Friends}
\label{sec:CaslAndFriends}

%%\subsection{\CASL}
%%\label{sec:casl}

The Common Algebraic Specification Language (\CASL) is a specification
language, which was developed by the \emph{Common Framework Initiative
  for Algebraic Specification and Development} (\CoFI).  \CASL allows
for writing algebraic specifications that can be expressed in a
many-sorted first order language with partial function symbols.  Basic
\CASL specifications consist of signature declarations and axioms
characterizing the models to be described.  These axioms, in turn, are
first-order formulae 
%% built from equations 
or assertions regarding the
definedness of partial function symbols.  Going beyond first-order
logic, \CASL also provides constructs to state induction principles
(called sort generation constraints) and datatype declarations.
Furthermore, specifications may contain subsort declarations, whereby
subsort inclusions are treated as embeddings.  Finally, \CASL also
provides constructs for structured specifications, namely,
translations, reductions, unions, and extensions of specifications.

In the sequel we will explain these concepts in more detail (for a full
discussion see %\citet{astesiano-a:02-casl} and
\citet{bidoit-b:04-casl} and \citet{CASL-RM}).


\subsection{Constructing Specifications}


To start with, let us briefly explain the formal underpinnings of
\CASL specifications.  As said before, \CASL allows for specifying
first order theories with partial function symbols. More precisely,
\CASL accepts languages with \emph{(many-sorted) signatures} $\Sigma =
\tupel{S,T\!F,P\!F,R}$ such that:
\begin{itemize}
\item $S$ is a (finite) set of sorts.
\item For each $(w,s)\in S^*\times S$, $T\!F_{w,s}$ and $P\!F_{w,s}$
  are disjoint sets of total and partial function symbols,
  respectively (tuples $w\in S^*$ are referred to as \emph{sort
    profiles}).
\item For each $w\in S^*$, $R_w$ is a set of relation symbols. 
\end{itemize}
As usual, individual symbols can be introduced as 0-ary total function
symbols. Accordingly, models of such signatures are many-sorted
partial first-order structures: Given a signature $\Sigma$, a
\emph{$\Sigma$-model} is a structure consisting of non-empty carrier
sets $s^M$ (for each sort $s\in S$), partial and total functions
$\klAbbb{f^M}{w^M}{s^M}$ (for each function symbol $f\in P\!F_{w,s}$
or $f\in T\!F_{w,s}$, respectively), and relations $r^M \subset w^M$
(for each relation symbol $r\in R_w$).


The most fundamental notion related to extensions, unions, and
translations of specifications is that of a signature morphism. To
explain this notion, let $\Sigma=\tupel{S,T\!F,P\!F,R}$ and
$\Sigma'=\tupel{S',T\!F',P\!F',R'}$ be signatures.  Then a
\emph{signature morphism} ${\Sigma}\rightarrow{\Sigma'}$ is a 4-tuple
$\sigma=\tupel{\sigma^{\rm s},\sigma^{\rm t},\sigma^{\rm
    p},\sigma^{\rm r}}$ consisting of maps (families of maps, resp.):
\begin{itemize}
\item $\klAbbb{\sigma^{\rm s}}{S}{S'}$,
\item $\klAbbb{\sigma^{\rm t}_{w,s}}{T\!F_{w,s}}{T\!F'_{\sigma^{\rm s}(w),\sigma^{\rm s}(s)}}$,
\item $\klAbbb{\sigma^{\rm
      p}_{w,s}}{P\!F_{w,s}}{T\!F'_{\sigma^{\rm s}(w),\sigma^{\rm s}(s)} \cup
    P\!F'_{\sigma^{\rm s}(w),\sigma^{\rm s}(s)}}$, and
\item $\klAbbb{\sigma^{\rm r}_{w}}{R_{w}}{R'_{\sigma^{\rm s}(w)}}$.
\end{itemize}
That is, partial function symbols may be mapped to total function
symbols, but not vice versa. 

  
On the semantic level, signature morphisms inherit models from the
target to the source signature.  To see this, let
$\klAbbb{\sigma}{\Sigma}{\Sigma'}$ be a signature morphism, and let
$M'$ be a $\Sigma'$-model. Then $\sigma$ defines a $\Sigma$-model
$M'|_{\sigma}$ (the \emph{$\sigma$-reduct} of $M'$) by
\begin{equation*}
  s^{M'|_{\sigma}} := \sigma^{\rm s}(s)^{M'}, \quad
  f^{M'|_{\sigma}} := \sigma^{\rm t/p}(f)^{M'}, \quad\text{and}\quad 
  r^{M'|_{\sigma}} := \sigma^{\rm r}(r)^{M'}. 
\end{equation*}
%%If $M=M'|_{\sigma}$, then $M'$ is called an \emph{expansion} of $M$,
%%and said to \emph{extend} $M$.

%%\smallskip

We are now ready to explain some fundamental notions in more detail. 
  
\paragraph{Basic specification.}
The most simple kind of \CASL specifications, called \emph{basic
  specifications}, are asserted by the keyword \textbf{spec} and have the
form
\begin{equation*}
  \textbf{spec}\ \textit{SpecName} = \text{Spec}
\end{equation*}
where \textit{SpecName} is the name of the specification and \text{Spec}
is a list of signature declarations and first order axioms. 

\paragraph{Extension.}
\emph{Extensions} have the form
\begin{equation*}
  \textit{Spec1}\ \textbf{then}\ \textit{Spec2}
\end{equation*}
where \textit{Spec1} is a specification or a specification name, and
\textit{Spec2} is a specification that extends the signature of
\textit{Spec1} and/or adds additional axioms.  
%% \textit{Spec2} will be referred to as the \emph{specification 
%% fragment} of an extension specification.  
\CASL allows to indicate the type of the extension: \emph{Definitional
  extensions} are introduced by annotating the keyword \textbf{then}
with \textbf{\%def}.  From a model-theoretical point of view,
definitional extensions are justified when each model of
\textit{Spec1} can be uniquely extended to a model of the
specification \textit{Spec1-then-Spec2}.  
%%\emph{Conservative
%%  extensions} are indicated by \textbf{then \%cons}, which means that
%%each model of \textit{Spec1} can be extended to a model of
%%\textit{Spec1-then-Spec2}.  To put it another way, conservative
%%extensions preserve satisfiability of the original specification.
\emph{Implied extensions} state some theorems of the original
specification, \ie\ \textit{Spec1} and \textit{Spec1-then-Spec2} have
the same model class.  This kind of extension is introduced by the
annotated keyword \textbf{then \%implies} and is considered
well-formed only if \textit{Spec1} and \textit{Spec1-then-Spec2} have
the same signature. 
  
\paragraph{Union.} 
It is possible to join two specifications, \ie\ to build their
\emph{union}.  The signature of a union of specifications
\textit{Spec1} and \textit{Spec2} is just the union of the signatures
of \textit{Spec1} and \textit{Spec2}. The models of the union are
exactly those models of the union signature whose reducts are models
of \textit{Spec1} and \textit{Spec2}, respectively.  Unions of
specifications obey the ``same name, same thing'' principle, which
means that each symbol contained in both signatures has a single
interpretation in each model of their union. Unions are declared by
\begin{equation*}
\textit{Spec1} \textbf{ and } \textit{Spec2}.
\end{equation*}

\paragraph{Translations.}
\emph{Translations} are renamings of sort symbols and/or signature
symbols and thus provide a signature morphism from the source
specification into the specification resulting from the translation.
The models of the translation are exactly those models of the result
specification whose reducts along the morphism are models of the
source specification.  Translations are declared by
\begin{equation*}
\textit{Spec} \textbf{ with } \text{SymbolMappings}.
\end{equation*}

\paragraph{Reductions.}  \CASL also provides constructs to restrict the
signature of a given specification. It is possible to hide symbols or,
alternatively, declare the symbols that are revealed.  Reductions can
be declared by 
\begin{equation*}
\textit{Spec} \textbf{ hide } \text{Symbols}\quad \text{and}\quad
\textit{Spec} \textbf{ reveal } \text{SymbolMappings}.
\end{equation*}



\paragraph{Parameterization and instantiations.}
\CASL also allows parameterized specifications, which are written as:
\begin{equation*}
\textbf{spec} \textit{ Spec1}[\textit{Spec2}]\dots[\textit{SpecN}] = \text{ Spec}.
\end{equation*}
Sometimes it is necessary to instantiate a previously declared
specification via a symbol mapping before it is used in a parameterized
specification. This can be obtained by
\begin{equation*}
\textit{Spec1}[ Spec2 \textbf{ fit } \text{SymbolMappings} ].
\end{equation*}


\paragraph{Views.}  A \emph{view} provides a signature morphism (defined by a symbol
mapping)
between two specifications. Actually, it is required to be a
\emph{theory morphism} (or \emph{interpretation of theories}), which
means that each model of the target specification induces (when reduced
via the signature morphism) a model of
the source specification. Views are declared by
\begin{equation*}
  \textbf{view} \textit{ View } \textbf{:} \textit{ Spec1 } \textbf{ to }
  \textit{ Spec2 } \textbf{=} \text{ SymbolMapping}.
\end{equation*}
The possibility of specifying views is one of the distinguished features of \CASL.




\subsection{\HasCASL}

\HasCASL (see \cite{HasCASL02}) is a higher-order extension of \CASL based on the partial
$\lambda$-calculus. The user declared sorts are used to generate
\emph{higher types} by closing the set of types under
total and partial function spaces (written by $t_1\rightarrow t_2$ 
and $t_1\rightarrow ? t_2$, resp.). Predicates (written $Pred ~t$)
are coded as partial functions into a singleton type (\ie\ only
the domain provides relevant information).
In fact, we often will need \HasCASL for specifying the model classes of 
qualitative calculi (e.\,g., for the real numbers, for metric and 
topological spaces).
\CASL's structuring constructs (union, translation, hiding, etc.) are
independent of the underlying logical system and hence can be used
for \HasCASL as well. 



\subsection{Tools}


\paragraph{\Hets.}
The \emph{Heterogeneous Tool Set (\Hets)} \citep{Hets}, which is
developed at the University of Bremen, Germany, is the main analysis
tool for \CASL and its extensions.  \Hets integrates a parser and a
type-checker for heterogeneous specifications.  A graphical interface
allows for presenting the development graph (showing the specification
structure) of \CASL specifications as well as the logic graph
presenting the underlying logic(s). It is possible to translate a
\CASL specification into XML-, \LaTeX-, and other formats. \Hets also
provides an interface to translate \CASL specifications into Isabelle
theory files. Of course, \Hets also supports \HasCASL specifications.




\paragraph{Isabelle.}
Isabelle is a generic proof assistant, which is developed by
L.~C.~Paulson (University of Cambridge, UK) and T.~Nipkow (Technical
University of Munich, Germany). Isabelle provides a rich language for
expressing mathematical formulae and contains tools for proving these
formulae in a logical calculus. Its main application fields are the
formalization of mathematical proofs and the formal verification of
computer languages, protocols, computer hardware, and software
specifications.

A main feature of Isabelle is that it is not restricted to a single
formal calculus since it supports higher-order logic, axiomatic set
theory, etc. We use Isabelle/HOL, the coding of higher-order logic in Isabelle.
%% Isabelle proofs can be written in two different ways: A
%% proof may consist of a sequence of proof commands (rule
%% applications), or can be written in a traditional mathematical proof
%% style (ISAR). 
%% Isabelle comes with a classical reasoner that 
%% enables tableau based reasoning in first order logic and with a
%% simplifier that solves equations by term rewriting. A user interface
%% to Isabelle is provided by (X)Emacs/Proof General. 
For a more comprehensive introduction we refer to \citet{nipkow-b:02-isabelle}.

% There exists a large number of axiomatized Isabelle theories and
% proofs from different mathematical fields such as elementary number
% theory, analysis (basic properties of limits, derivatives and
% integrals), algebra (Sylow's theorem), and set theory (the relative
% consistency of the Axiom of Choice). 


\section{Specifications of Qualitative Calculi}

\subsection{Relation Algebras}

To start with, let us first discuss some specifications related to
relation algebras.  A \emph{relation algebra} is a Boolean algebra
with complement (its elements are referred to as \emph{relations})
and with a distinguished element $\OPid$ (the \emph{identity
  relation}), a unary total function $\inv{}$ assigning to each
relation its converse relation (algebraically, its \emph{involution}),
and a binary total function $\cmps$ assigning to each pair of
relations their composition. Note that here the term ``relation'' is
used in an abstract manner, \ie\ independently of its usual
set-theoretical interpretation. Rather, these functions are
characterized implicitly by the axioms listed in the following
specification.
\begin{lstlisting}
spec RelationAlgebra  =
     BooleanAlgebraWithCompl with sort $\OPElem \mapsto \OPRel$ 
then
     ops $\OPid$ : $\OPRel$;
         __$\inv{}$ : $\OPRel \rightarrow \OPRel$;
         __$\cmps$__ : $\OPRel \times \OPRel \rightarrow \OPRel$, assoc, unit $\OPid$;
     $\forall x,y,z:\,\OPRel$
     $\casldot$ $\inv{(\inv x)} = x$                                             @\hfill@ %(inv_idempot)%
     $\casldot$ $\inv{(x\sqcup y)} = \inv x\sqcup \inv y$                           @\hfill@ %(inv_cup)%
     $\casldot$ $\inv{(\compl x)} = \compl{\inv x}$                              @\hfill@ %(inv_compl)%
     $\casldot$ $\inv{(x\cmps y)} = \inv y\cmps \inv x$                         @\hfill@ %(inv_cmps)%
     $\casldot$ $(x\cmps y)\sqcap\inv z = 0 \Rightarrow (y\cmps z)\sqcap \inv x = 0$ @\hfill@ %(triangle)% 
then %implies
     $\forall x,y,z:\, \OPRel$
     $\casldot$ $(x\sqcup y)\cmps z = (x\cmps z)\sqcup (y\cmps z)$          @\hfill@ %(cmps_cup_rdistrib)%
     $\casldot$ $z\cmps (x\sqcup y) = (z\cmps x)\sqcup (z\cmps y)$          @\hfill@ %(cmps_cup_ldistrib)%
     $\casldot$ $(\inv x\cmps \compl{(x\cmps y)})\sqcap y = 0$   @\hfill@ %(RelAlg)%
end
\end{lstlisting}

We may define a partial order on a relation algebra in exactly the
same way as we could introduce it for arbitrary Boolean algebras. This
gives us some nice corollaries, which can easily be proven by
Isabelle. For example, the composition of relations behaves
monotonic with respect to the canonical partial order.

\begin{lstlisting}
spec ExtRelationAlgebraByPartialOrder[RelationAlgebra] = %def
     ExtBooleanAlgebraByPartialOrder[BooleanAlgebra] 
     with sort $\OPElem \mapsto \OPRel$
then %implies
     $\forall x,y,x',y':\, \OPRel$
     $\casldot$ $x \leq x' \land y \leq y' \Rightarrow x\cmps y \leq x'\cmps y'$                @\hfill@ %(cmps_monotonic)%
     $\casldot$ $x \leq \OPid \Rightarrow \inv x = x$                                     @\hfill@ %(inv_below_id)%
end
\end{lstlisting}

In the following, we will be mainly interested in \emph{atomic relation
  algebras}. An \emph{atom} of a Boolean algebra is a non-zero element
$x$ such that the zero element is the only element $y$ with $y<x$.
A relation algebra (or Boolean algebra) is said to be \emph{atomic} if 
for each non-zero element $x$, there exists an atom $a$ with
$a\leq x$.  Hence the unary predicate ``Atom'' defines a genuine
subsort.

\begin{lstlisting}
spec AtomicRelationAlgebra =  
     RelationAlgebra 
and  AtomicBooleanAlgebra with sort $\OPElem \mapsto \OPRel$, $\OPAtomElem \mapsto \OPAtomRel$
end
\end{lstlisting}

As explained in section~2, often an (abstract) atomic relation algebra
can be constructed from a set of base relations and a composition
table.  In \CASL this procedure can be reconstructed as follows: First
we define relations (\ie\ sort $\OPRel$) as arbitrary sets of base
relations such that base relations correspond to singleton sets of
base relations.

\bigbreak

\begin{lstlisting}
spec SetRepresentationOfRelations [sort $\OPBaseRel$] = %def
local { Set [sort $\OPBaseRel$ fit $\OPElem$ $\mapsto$ $\OPBaseRel$] 
        with __$\union$__ $\mapsto$ __$\sqcup$__, __$\intersection$__ $\mapsto$ __$\sqcap$__, __$\subset$__ $\mapsto$ __$\sqsubset$__ }
within
     free type $\OPRel$ ::= sort $Set[\OPBaseRel]$ 
     sort $\OPBaseRel < \OPRel$
     ops  $0,1$ : $\OPRel$;
          $\compl$   : $\OPRel\rightarrow \OPRel$;        
          __$\sqcup$__, __$\sqcap$__: $\OPRel \times \OPRel \rightarrow \OPRel$    
     preds __$\eps$__: $\OPBaseRel \times \OPRel$;
          __$\sqsubset$__ : $\OPRel \times \OPRel$
     $\forall x\,:\,\OPBaseRel;\; r\,:\,\OPRel$
     $\casldot$ $x = \{x\}$
     $\casldot$ $x\eps 1 \land \neg x\eps 0$
     $\casldot$ $x\eps\compl{r} \Leftrightarrow \neg x \eps r$
then %implies
     $\dots$
end

view $\SpecName{SetRepresentation\_as\_AtomicBooleanAlgebra}$ [sort $\OPBaseRel$] :
     AtomicBooleanAlgebra 
to   SetRepresentationOfRelations [sort $\OPBaseRel$]
=    $\OPElem \mapsto \OPRel$, $\OPAtomElem \mapsto \OPBaseRel$
end
\end{lstlisting}

In a second step we generate a relation algebra-like structure by
extending the functions ``composition'' and ``involution'' (as defined
for base relations) to total functions on all relations.  In fact, not
each composition table defines a relation algebra since many such
constructed structures violate the associativity axiom of relation
algebras \citep[see, e.\,g.,][]{ligozatR-a:04-what}. The view
contained in the following small library states that we obtain a
genuine relation algebra if the composition table satisfies certain
conditions. For the sake of simplicity, our specification of
composition tables includes both the the composition
function and the involution function for base relations.


\begin{lstlisting}
spec CompositionTable = 
     sorts $\OPBaseRel < \OPRel$
     ops $\OPid$ : $\OPBaseRel$;
         $0,1$ : $\OPRel$;          
         __$\inv{}$ : $\OPBaseRel \rightarrow \OPBaseRel$;
         __$\cmps$__ : $\OPBaseRel \times \OPBaseRel \rightarrow \OPRel$;
         $\compl$__  : $\OPRel \rightarrow \OPRel$;   
         __$\sqcup$__ : $\OPRel \times \OPRel \rightarrow \OPRel$, assoc, idem, comm, unit 1
     $\forall x:\,\OPBaseRel$
     $\casldot$ $x \cmps \OPid = x \;\land\; \OPid \cmps x = x$
     $\casldot$ $\inv{\OPid} = \OPid$
     $\casldot$ $\inv{(\inv{x})} = x$ 
end
\end{lstlisting}

\bigbreak\bigbreak

\begin{lstlisting}
spec ConstructRelationAlgebra [sort $\OPBaseRel$] [CompositionTable] = %def
     SetRepresentationOfRelations [sort $\OPBaseRel$]
then %def
     ops $\OPid$ : $\OPRel$;
         __$\inv{}$ : $\OPRel \rightarrow \OPRel$;
         __$\cmps$__ : $\OPRel \times \OPRel \rightarrow \OPRel$;
     $\forall x,y:\,\OPBaseRel;\, r,s:\,\OPRel$
     $\casldot$ $x \eps \inv{r} \Leftrightarrow \inv{x} \eps r$
     $\casldot$ $x \eps (r \cmps s) \Leftrightarrow \exists y,z:\OPBaseRel \casldot y \eps r \land z \eps s \land x \eps (y \cmps z)$
then %implies
     op __$\cmps$__ : $\OPRel \times \OPRel \rightarrow \OPRel$, unit $\OPid$;       
end
\end{lstlisting}


\begin{lstlisting}
view $\SpecName{ConstructedRelationAlgebra\_as\_AtomicRelationAlgebra}$ 
            [sort $\OPBaseRel$] [GoodCompositionTable]:
     AtomicRelationAlgebra 
to   ConstructRelationAlgebra [sort $\OPBaseRel$] [GoodCompositionTable] 
=    $\OPRel \mapsto \OPRel$, $\OPAtomRel \mapsto \OPBaseRel$
end
\end{lstlisting}



Let us now turn to the semantic level. First we define the concept
\emph{algebra of binary relations (BRA)}.  Given a set $X$, an algebra
of binary relations is a Boolean subalgebra of the set algebra of all
binary relations on $X$ that contains the identity relation and is
closed with respect to involution and composition (in their
usual set-theoretical meaning). Of course, each algebra of binary
relations is a relation algebra. But contrary to Boolean algebras
(cf.\ Stone's representation theorem), it is in general not the case
that each relation algebra can be represented as an algebra of binary
relations.
\begin{lstlisting}
logic HasCASL

spec BinaryRelations [sort $\OPElem$] = %mono 
     Set        
then %mono 
     type $\OPRelation$ ::= abs$(\OPrep:Set(\OPElem\times \OPElem))$
end

spec SetAlgebraOfBinaryRelations =
     BinaryRelations [sort $\OPElem$]
then
     type $\OPRel < \OPRelation$
     ops $0,1$ : $\OPRel$;
         $\compl$ : $\OPRel \rightarrow \OPRel$;
         __$\sqcup$__, __$\sqcap$__ : $\OPRel \times \OPRel \rightarrow \OPRel$
     $\forall r,s:\,\OPRel$
     $\casldot$ $\OPrep(0) = \emptySet$
     $\casldot$ $\OPrep(1) = \name{allSet}$
     $\casldot$ $\OPrep(r \sqcup s) = \OPrep(r) \union \OPrep(s)$
     $\casldot$ $\OPrep(r \sqcap s) = \OPrep(r) \intersection \OPrep(s)$
     $\casldot$ $\OPrep(\compl r) = \OPrep(1) \setminus \OPrep(r)$
end
\end{lstlisting}

\bigbreak

\begin{lstlisting}
spec AlgebraOfBinaryRelations  =
     SetAlgebraOfBinaryRelations 
then
     ops $\OPid$ : $\OPRel$;
         __$\inv{}$ : $\OPRel \rightarrow \OPRel$;
         __$\cmps$__ : $\OPRel \times \OPRel \rightarrow \OPRel$;
     $\forall r,s:\,\OPRel;\; x,y:\,\OPElem$
     $\casldot$ $(x,y)\isIn \OPrep(r \cmps s) \Leftrightarrow \exists z:\,\OPElem \casldot (x,z)\isIn \OPrep(r) \land (z,y)\isIn \OPrep(s)$ 
     $\casldot$ $(x,y)\isIn \OPrep(\inv r) \Leftrightarrow (y,x)\isIn \OPrep(r)$
     $\casldot$ $(x,y)\isIn \OPrep(\OPid) \Leftrightarrow x = y$
then %implies 
     ops __$\cmps$__ : $\OPRel\times \OPRel \rightarrow \OPRel$, assoc, unit $\OPid$
end
\end{lstlisting}

\begin{lstlisting}
spec FullAlgebraOfBinaryRelations [sort $\OPElem$] = %def  
     { AlgebraOfBinaryRelations with type $\OPRel \mapsto \OPRelation$ }
end
\end{lstlisting}

\begin{lstlisting}
view $\SpecName{AlgebraOfBinaryRelations\_as\_RelationAlgebra}$:
     RelationAlgebra to AlgebraOfBinaryRelations 
end
\end{lstlisting}

In the following, we decribe how a strong representation of an (abstract)
atomic relation algebra can be constructed from a concrete
interpretation of its atoms, \ie\ its base relations. For this assume
that we have a model for the base relations of the relation algebra,
that is, a JEPD system of relations on a non-void set $\OPElem$. Of
course, we want to extend this model in a canonical manner to a model
of all relations.  Relations are here exactly those binary relations
on $\OPElem$ that can be written as a (set-theoretical) union of base
relations. In order to be a strong representation, the model we aim at
must be an algebra of binary relations. But the fact that the system
of relations is JEPD ensures only that the set of all unions of base
relations forms a set algebra. This set algebra need not be an algebra
of binary relations since, in general, it need not contain the identity
relation, be closed with respect to involution, or closed with respect
to composition.  But if it does\,---\,a base relation model satisfying
these conditions will be referred to as \emph{closed for
  composition}\,---\,the canonical extension of the base relation
model defines an atomic algebra of binary relations. In fact, the
final view in the following list of specifications states that the
concrete relation algebra defined in this way provides a strong
representation of the abstract algebra.

\begin{lstlisting}
spec JEPDBaseRelModel =
     BinaryRelations [sort $\OPElem$]       
then 
     type $\OPBaseRel < \OPRelation$
     $\forall x,y:\,\OPElem;\; r,s:\,\OPBaseRel$
     $\casldot$ $\exists r:\OPBaseRel \casldot (x,y) \isIn \OPrep(r)$        @\hfill@ %(JointlyExhaustive)%
     $\casldot$ $\neg r = s \Rightarrow \OPrep(r) \intersection \OPrep(s) = \emptySet$ @\hfill@ %(PairwiseDisjoint)%
end
\end{lstlisting}

\bigbreak

\begin{lstlisting}
spec RelationsFromBaseRelModel [JEPDBaseRelModel] = %def
     type $\OPRel$ = $\{x:\OPRelation \casldot \exists X:Set(\OPBaseRel) \casldot $  
         $(\forall y,z:\OPElem \casldot (y,z) \isIn \OPrep(x)\Leftrightarrow  (\exists r:\,\OPBaseRel \casldot r \isIn X \land (y,z) \isIn \OPrep(r)))\}$
end
\end{lstlisting}

\begin{lstlisting}
spec ConstructModel [CompClosedBaseRelModel] = %def
     RelationsFromBaseRelModel [JEPDBaseRelModel] 
and  AlgebraOfBinaryRelations
then %def 
     preds __$\eps$__ : $\OPBaseRel * \OPRel$;
           __$\sqsubset$__ : $\OPRel * \OPRel$
     $\forall x\,:\,\OPBaseRel;\, r,r'\,:\,\OPRel$
     $\casldot$ $x \eps r \Leftrightarrow \OPrep(x) \subset \OPrep(r)$
     $\casldot$ $r \sqsubset r' \Leftrightarrow \OPrep(r) \subset \OPrep(r')$
end
\end{lstlisting}

\begin{lstlisting}
view $\SpecName{RelationAlgebra\_from\_BaseRelModel}$ [CompClosedBaseRelModel] :
     RelationAlgebra 
to   ConstructModel [CompClosedBaseRelModel]
end
\end{lstlisting}


\subsection{RCC5, RCC8, and Allen's Interval Algebra}

We are now ready to explain how spatial and temporal calculi (more
exactly, their respective relation algebras) can be incorporated into
the framework provided by the specifications presented in the previous
section. Let us start with the region connection calculi RCC5 and
RCC8.  Both calculi describe relations between \emph{regions}, which
may be thought of as non-void, regular open (or alternatively, regular
closed) subsets of a topological space.  In RCC5 the following
relations count as base relations: $\rccDR$ (``discrete''), $\rccPO$
(``partially overlap''), $\rccPP$ (``proper part''), $\rccPPi$
(the converse of $\rccPP$), and $\rccEQ$ (``equal'').  The set of RCC8
base relations is more fine-grained: $\rccDR$ splits into the
relations $\rccDC$ (``disconnected'') and $\rccEC$ (``externally
connected'') and $\rccPP$ into the relations $\rccTPP$ (``tangential
proper part'') and $\rccNTPP$ (``non-tangential proper part'').  These
relations can be defined in terms of the topological closure
operation: the relation $\rccDC$ holds between regular open sets $X$
and $Y$ if their closures do not intersect, $X\, \rccNTPP\,\,Y$ holds if
the closure of $X$ is contained in $Y$, etc.

\smallskip

The following library %%presented in Spec.~\ref{spec:RCC5}
shows how the (abstract) relation algebra of RCC5 can be defined via
\CASL\ specifications:
%%\begin{spec}[p]
\begin{lstlisting}
spec RCC5BaseRelations = %mono
     free type $\OPBaseRel$ ::= $\OPdr\mid \OPpo\mid \OPpp\mid \OPppi\mid \OPeq$
end
\end{lstlisting}

\begin{lstlisting}
spec RCC5CompositionTable =
     sort $\OPBaseRel$
     ops $\OPdr,\OPpo,\OPpp,\OPppi,\OPeq$: $\OPBaseRel$
and  CompositionTable with op $\OPid \mapsto \OPeq$
then
     $\casldot$ $\inv{\OPdr} = \OPdr$                                @\hfill@ %(sym_dr)%
     $\casldot$ $\inv{\OPpo} = \OPpo$                                @\hfill@ %(sym_po)%
     $\casldot$ $\inv{\OPpp} = \OPppi$                               @\hfill@ %(inv_pp)%
     $\casldot$ $\inv{\OPppi} = \OPpp$                               @\hfill@ %(inv_ppi)%
     $\casldot$ $\OPpp \cmps \OPpp  = \OPpp$                         @\hfill@ %(cmps_pppp)%  
     $\casldot$ $\OPpp \cmps \OPppi = 1$                             @\hfill@ %(cmps_ppppi)%
     $\casldot$ $\OPpp \cmps \OPpo  = \OPpp \sqcup \OPpo \sqcup \OPdr$  @\hfill@ %(cmps_pppo)%
     $\casldot$ $\OPpp \cmps \OPdr  = \OPdr$                         @\hfill@ %(cmps_ppdr)%
     $\casldot$ $\OPppi \cmps \OPpp  = \compl{\OPdr}$                @\hfill@ %(cmps_ppipp)% 
     $\casldot$ $\OPppi \cmps \OPppi = \OPppi$                       @\hfill@ %(cmps_ppippi)%
     $\;\dots$
end
\end{lstlisting}

\begin{lstlisting}
spec RCC5 = 
     ConstructRelationAlgebra [RCC5BaseRelations]
        [RCC5CompositionTable fit op $\OPid$:$\OPBaseRel \rightarrow  \OPeq$]
end
\end{lstlisting}

\begin{lstlisting}
view $\SpecName{RCC5\_as\_AtomicRelationAlgebra}$ :
     AtomicRelationAlgebra to RCC5      
=    $\OPAtomRel \mapsto \OPBaseRel$
end
\end{lstlisting}
%% \caption{The specification of RCC5}
%% \label{spec:RCC5}
%% \end{spec}
%% The main step is done in the specification
%% \textsc{RCC5CompositionTable}, in which we encode both the
%% composition table of RCC5 and the involution function restricted to base
%% relations.  

Obviously, the relation algebras of other qualitative
calculi (such as RCC8 or Allen's interval algebra) can be specified in
the very same manner. Moreover, we can declare natural views between
these calculi as follows:
 
\begin{lstlisting}
view $\SpecName{RCC5\_to\_RCC8}$ :
     RCC5 
to   { RCC8 then %def
       ops $\OPdr,\OPpp,\OPppi$ : $\OPRel$
       $\casldot$ $\OPdr = \OPdc \sqcup \OPec$
       $\casldot$ $\OPpp = \OPtpp \sqcup \OPntpp$
       $\casldot$ $\OPppi = \OPtppi \sqcup \OPntppi$ }
=    sort $\OPBaseRel \mapsto \OPRel$ 
end
\end{lstlisting}

\begin{lstlisting}
view $\SpecName{RCC5\_to\_AllenIA}$ :
     RCC5 
to   { AllenIA then %def
       ops $\OPdr,\OPpo,\OPpp,\OPppi,\OPeq$ : $\OPRel$
       $\casldot$ $\OPdr = \OPb \sqcup \OPbi \sqcup \OPm \sqcup \OPmi$  @\hfill@ %[ Allen relations are denoted by the first ]%
       $\casldot$ $\OPpo = \OPo \sqcup \OPoi$        @\hfill@ %[ letter of their names (cf. Fig. 1), i.e., ]%
       $\casldot$ $\OPpp = \OPd \sqcup\OPs \sqcup \OPf$        @\hfill@ %[ b: before, bi: before involuted, etc.    ]%         
       $\casldot$ $\OPppi = \OPdi\sqcup\OPsi \sqcup \OPfi$
       $\casldot$ $\OPeq = \OPe$ }
=    sort $\OPBaseRel \mapsto \OPRel$ 
end
\end{lstlisting}

It is worth mentioning that a corresponding view from
$\SpecName{RCC8}$ to $\SpecName{AllenIA}$ is not valid: The function
$\OPdc \mapsto \OPb \sqcup \OPbi$, $\OPec \mapsto \OPm \sqcup \OPmi$,
$\OPpo \mapsto \OPo \sqcup \OPoi$, $\OPtpp \mapsto \OPs \sqcup \OPf$,
$\OPntpp \mapsto \OPd$, etc., only defies a view from RCC8 to Allen's
interval algebra \emph{as} Boolean algebras, but not as relation
algebras. To put it another way, RCC8 does not form a subalgebra of
Allen's interval algebra. 

\smallskip


Let us now illustrate how the semantic level of such relation algebras
can be presented via \CASL\ specifications. The following library
describes a specific class of RCC5 models, which is definable for the
Euclidean plane (interpreted as a metric space).  Its final view says
that one obtains a model of RCC5 if we interpret the RCC5 base
relations over open discs in the Euclidean plane. More precisely, it
states that the canonical interpretation of RCC5 over open discs in
the Euclidean plane provides a strong representation of RCC5.%
\footnote{%
  \emph{Weak} representations of abstract
  relation algebras could be presented as \CASL specifications as
  well, but not in terms of \CASL views.}

\begin{lstlisting}
logic HasCASL

view $\SpecName{EuclideanPlane\_as\_MetricSpace}$ :
     MetricSpace to EuclideanPlane
     $\dots$
end
\end{lstlisting}

\begin{lstlisting}
spec RCC5OpenDiscBaseRelModel[MetricSpace] = 
     Set
then op $\name{openDisc}(r:\name{Real};x:\name{Elem})$ : $\name{Set}\,\,\name{Elem} = \lambda y:\name{Elem} \casldot \name{dist}(x,y)<r$
     type $\name{OpenDisc} = \{X:\,\name{Set}\, \name{Elem} \casldot \exists r:\,\name{Real};\ x:\,\name{Elem} \casldot X=\name{openDisc}(r,x)\}$
then BinaryRelations [sort $\name{OpenDisc}$] 
then %def
     ops $\OPdrRel,\OPpoRel,\OPppRel,\OPppiRel,\OPeqRel:\OPRelation$ 
     type $\OPBaseRel$ ::= $\OPppRel\mid \OPppiRel\mid \OPpoRel\mid \OPdrRel\mid \OPeqRel$ 
     $\forall x,y:\,\name{OpenDisc}$
     $\casldot$ $(x,y)\isIn \OPrep(\OPdrRel) \Leftrightarrow x~ \name{disjoint}~ y $
     $\casldot$ $(x,y)\isIn \OPrep(\OPpoRel) \Leftrightarrow \neg \, x \subset y \land \neg y \subset x \land \neg\, x~ \name{disjoint}~ y$
     $\casldot$ $(x,y)\isIn \OPrep(\OPppRel) \Leftrightarrow x \subset y \land \neg x = y$
     $\casldot$ $(x,y)\isIn \OPrep(\OPppiRel) \Leftrightarrow y \subset x \land \neg x = y$
     $\casldot$ $ (x,y)\isIn \OPrep(\OPeqRel) \Leftrightarrow x = y$
end
\end{lstlisting}

\begin{lstlisting}
spec RCC5OpenDiscModel[EuclideanPlane] = %def
     ConstructModel [RCC5OpenDiscBaseRelModel [
          view $\SpecName{EuclideanPlane\_as\_MetricSpace}$] 
        fit sort $\OPElem \mapsto \name{OpenDisc}$] 
end
\end{lstlisting}

\begin{lstlisting}
view $\SpecName{MetricSpace\_induces\_RCC5OpenDiscModel}$ :
     RCC5 
to   RCC5OpenDiscModel [EuclideanPlane]
=    ops $\OPpp\mapsto \OPppRel$, $\OPppi\mapsto \OPppiRel$, $\OPpo\mapsto \OPpoRel$, $\OPdr\mapsto \OPdrRel$, $\OPeq\mapsto \OPeqRel$ 
end
\end{lstlisting}

The development graph of these libraries, as output by \Hets\ (a subgraph
of it is depicted in Fig.~\ref{fig:hetsOutput}), 
\begin{figure}[ht!]
  \begin{center}
    \includegraphics[width=122mm,height=75mm]{graphics/graph2}
  \end{center}
  \caption{Development graph of RCC5, RCC8, and Allen's interval algebra}
  \label{fig:hetsOutput}
\end{figure}
exhibits the mutual dependencies between the presented specifications.
Dark arrows denote inclusions of theories (a double arrow denotes
that such an inclusion is across different logics); light arrows denote proof
obligations (theory morphisms) generated by views.
The complete set of specifications is available under \url{http://www.cofi.info/Libraries}.




  
\section{Summary and Outlook}

In this paper we discussed how qualitative calculi can be described
via \CASL specifications. We saw that \CASL allows for an elegant
representation of such calculi. Moreover, since the specifications
presented here are built up in a modular way, we provide an easy
interface for embedding other calculi into the \CASL framework.
Finally, \CASL specifications ensure a high visibility of the mutual
dependencies between qualitative calculi on both the syntactic and the
semantic level, and hence may be considered superior to ontologies of
such systems.

Because of lack of space, we could not show how these \CASL
specifications connect to theorem provers such as Isabelle.
Furthermore, in this paper we could only explain a small fraction of
algebraic theories related to spatial and temporal calculi. For
example, the first-order theories of these calculi and Stell's
Boolean connection algebras \citep{stell-a:00-boolean} should be
integrated into \CASL as well.  Future work will also deal with the CSP
languages of these qualitative calculi. In particular, we will
investigate how these languages can be translated into \ModalCASL, a
sublanguage of \CASL designed for specifying multi modal fragments of
first order logic.  Such translations would be particularly
interesting for automated verifications of composition tables of the
calculi mentioned in this paper.

\section*{Acknowledgments}

This work was partially supported by the Deutsche
Forschungsgemeinschaft (DFG) as part of the Transregional
Collaborative Research Center SFB/TR\,8 Spatial Cognition.  We would
like to thank Klaus L�ttich and Bernhard Nebel for helpful
discussions. We also gratefully acknowledge the reviewers' critical
comments, as well as their hints and suggestions.


{\small
\bibliography{references}}

 
\end{document}



%%% Local Variables: 
%%% mode: PDF Latex
%%% TeX-master: t
%%% End: 
